%
%   Образец / Шаблон оформления тезиса
%
%
%   Если в тезисе каких-то разделов (картинок, списка литературы) нет, то соотвествующие команды надо закомментировать.
%   Файл для компиляции --- этот (example.tex, переименовый в фамилию автора, например, ivanov.tex).
%
%   ========================================================================================
%


%
%   Если в вашем документе нет картинок и вы хотите компилировать документ при помощи latex->dvips->ps2pdf, то уберите опцию usePics, заменив следующую строчку на
%\documentclass{lomonosov}
\documentclass[usePics]{lomonosov}

\begin{thesis}  % Сам тезис должен быть полностью помещен внутри окружения thesis

% Один автор
%\Title{Тема доклада}{{Иванов\,И.\,И.}}
% Несколько авторов
\Title{Некоторые способы применения перегоревших ламп накаливания}{{Ильин\,А.\,В.}{Шевцова\,И.\,Г.}}

%
%   Команда авторства. Выберете ту, что отвечает вашему тезису, и, если надо, раскомментируйте ее; остальные --- удалите или закомментируйте.
%

% Один автор
%\Author{Иванов~Иван~Иванович}{Студент}{Факультет ВМК МГУ имени М.\,В.\,Ломоносова}{Москва}{Россия}{ivanov@cmc.msu.ru}

% Несколько авторв из одной организации
\Author{Ильин Александр Владимирович, Шевцова Ирина Геннадьевна}{Математик, ассистент}{Факультет ВМК МГУ имени М.\,В.\,Ломоносова}{Москва}{Россия}{smu@cs.msu.ru, lomonosov@cs.msu.ru}

% Несколько авторов из разных организаций
%\AuthorM{{Иванов~Иван~Иванович}{Петров~Петр~Петрович}}{%
%   {Аспирант, факультет ВМК МГУ имени М.\,В.\,Ломоносова, Москва, Россия}{Младший научный сотрудник, Ленинградский кораблестроительный институт, Ленинград, СССР}}{ivanov@cmc.msu.ru, petrov@cmc.msu.su}

Лампы накаливания остаются сегодня широко распространенным
источником света, который используется как в быту, так и в
общественных и промышленных зданиях и сооружениях. Конструктивно
электрическая лампа накаливания состоит из цоколя, контактных
проводников и стеклянной колбы, ограждающей нить накала от
окружающей среды. Принцип работы ламп накаливания основан на
оптическом излучении, получаемом от разогретого до высокой
температуры проводника, который находится в инертной среде.
Критикуют электрические лампы накаливания за относительно короткий
срок службы (1000--1500 часов), а также за то, что более 95\%
энергии в них преобразуется в тепло и только 5\% --- в свет. В
данной работе мы показываем, что перечисленные недостатки являются
несущественными и могут быть успешно использованы в других целях.
Например, электрическая лампа накаливания является оптимальным
одновременным источником тепла и света в куриных инкубаторах~[3].
Достоверно известно, что перегоревшие лампы накаливания подлежат
ремонту, легко осуществимому в домашних условиях: для этого просто
нагрейте перегоревшую лампочку до температуры 900 градусов, затем
аккуратно снимите цоколь, замените нить накаливания, откачайте
воздух из колбы лампы и заполните ее инертным газом. Теперь самое
время сделать небольшой перерыв на чай или кофе! Не забудьте съесть
плиточку шоколада: он придаст вам сил! По окончании перерыва снова
нагрейте лампу до 900 градусов и поставьте цоколь на место.
Отремонтированная лампа будет служить долго и надежно!

Мы также показываем, что электрические лампы накаливания могут
использоваться не только как источники тепла или света~[2], но и для
достижения других, часто более важных и полезных целей.
Действительно, лампочка может служить мишенью для стрельбы из
пневматики, источником вольфрама для химических опытов,
линзой/увеличительным стеклом, или даже термометром: для этого
необходимо провертеть в ней дырочку, залить наполовину водой,
повесить за окно и по каплям конденсата по утрам предсказывать
погоду.

\begin{theorem}\label{iline-shevtsova-th1}
Лампа накаливания --- замечательная подарочная упаковка для живого
растения!
\end{theorem}

%\begin{proof}
Действительно, для того чтобы вырастить в лампочке цветок и
убедиться в справедливости теоремы~\ref{iline-shevtsova-th1},
достаточно снять цоколь, удалить перегоревшую нить накаливания и
смонтировать систему, изображенную на рис.\,1 слева. Цифрами
обозначены: 1~---~вентиляционная трубка (её можно затыкать снаружи
на некоторое время после полива, чтобы препятствовать испарению),
2~---~трубка для полива путём впрыскивания через нижнее отверстие
питательной жидкости с помощью шприца или резиновой груши,
3~---~собственно, растение, 4~---~колба, 5~---~цоколь, 6~---~грунт.
%\end{proof}

Следует заметить, что лампочка~--- отличное украшение на новогоднюю
ёлку, пусть даже и перегоревшая. Кроме того, её можно использовать
для штопки носков, или сбросить с крыши высокого-высокого здания, с
наслаждением отсчитывая секунды до хлопка. Зная длительность падения
$t$, можно вычислить высоту здания $h$ по формуле:
\begin{equation}\label{iline-shevtsova-eq1}
h=\frac{at^2}{2}
\end{equation}
где $a$~--- ускорение свободного падения в воздухе (часто в
формуле~\eqref{iline-shevtsova-eq1} берут $a$ равным $9.8$ м/с$^2$).
Известны также случаи успешного засовывания лампочек в рот на
спор~[1].

Садоводы--любители могут вырастить в лампе накаливания сувенирный
огурец особой формы. Для этого достаточно просто надеть пустую колбу
на огурец, находящийся в зародышевом состоянии! Последовательные
этапы развития Огурца зеленого сувенирного формы лампообразной
изображены на рис.\,1 справа.


При наличии набора перегоревших лампочек разной мощности и размера
можно соорудить музыкальный инструмент. Не следует забывать, что
перегоревшая лампочка~--- отличный повод одинокой женщине
познакомиться с соседом мужчиной! Кроме того, стеклянная колба с
аккуратно обработанным краями будет служить удивительно удобным
бокалом для тех, кто любит пить до дна.


Наконец, можно собрать самую большую в мире коллекцию перегоревших
лампочек и соорудить из неё мемориал с музеем П.\,Н.\,Яблочкову и
А.\,Н.\,Лодыгину~--- русским изобретателям ламп накаливания!


\Pictures
%Следующая команда повторяется для каждой иллюстрации
\Picture{iline-shevtsova}{Рис.\,1. Слева: чертеж подарочной
упаковки для живого растения. Справа: последовательные этапы
развития Огурца зеленого сувенирного формы лампообразной}{0.8}


В заключение авторы выражают признательность участникам Форума
объединенных сетей МГУ \url{http://forumbgz.ru} за тематическую
идею, профессору кафедры математической статистики факультета ВМК
МГУ д.ф.-м.н. В.\,Ю.\,Королёву за любовь к живой природе и
профессору кафедры НДСиПУ факультета ВМК МГУ, д.ф.-м.н.
В.\,В.\,Фомичёву за техническое оснащение и помощь в проведении
опытов.

Работа поддержана Министерством просвещения, гранты 14-05-00378 и
14-07-00945.

%
%   Список литературы, если он есть
%
\begin{references}
\Source Задорнов\,М.\,Н. Дорогая лампочка. М.:~Эксмо, 1990.

\Source  Лодыгин\,А.\,Н. О способах добывания электрической энергии
в Олонецкой и Нижегородской губерниях для пользования ею в местных
кустарных промыслах. Санкт-Петербург, 1914.

\Source  Ряба\,К.\,И. Использование ламп накаливания в инкубаторах.
Куриные хроники. 2003. Т.\,23, \No\,2. C.123--166.

\end{references}

\end{thesis} % Сам тезис должен быть полностью помещен внутри окружения thesis
