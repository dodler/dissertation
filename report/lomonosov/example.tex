%
%   Образец / Шаблон оформления тезиса
%
%
%   Если в тезисе каких-то разделов (картинок, списка литературы) нет, то соотвествующие команды надо закомментировать.
%   Файл для компиляции --- этот (example.tex, переименовый в фамилию автора, например, ivanov.tex).
%
%   ========================================================================================
%


%
%	Если в вашем документе нет картинок и вы хотите компилировать документ при помощи latex->dvips->ps2pdf, то уберите опцию usePics, заменив следующую строчку на
%\documentclass{lomonosov}
\documentclass[usePics]{lomonosov}
 
\begin{thesis}  % Сам тезис должен быть полностью помещен внутри окружения thesis
  
% Один автор
\Title{Тема доклада}{{Иванов\,И.\,И.}} 
% Несколько авторов
%\Title{Тема доклада}{{Иванов\,И.\,И.}{Петров\,П.\,П.}{Сидоров\,С.\,С.}} 

%
%	Команда авторства. Выберете ту, что отвечает вашему тезису, и, если надо, раскомментируйте ее; остальные --- удалите или закомментируйте. 
%

% Один автор 
\Author{Иванов~Иван~Иванович}{Студент}{Факультет ВМК МГУ имени М.\,В.\,Ломоносова}{Москва}{Россия}{ivanov@cmc.msu.ru}

% Несколько авторв из одной организации
%\Author{Иванов~Иван~Иванович, Петров~Петр~Петрович}{Студент, аспирант}{Факультет ВМК МГУ имени М.\,В.\,Ломоносова}{Москва}{Россия}{ivanov@cmc.msu.ru, petrov@lki.su}

% Несколько авторов из разных организаций
%\AuthorM{{Иванов~Иван~Иванович}{Петров~Петр~Петрович}}{%
%	{Аспирант, факультет ВМК МГУ имени М.\,В.\,Ломоносова, Москва, Россия}{Старший научный сотрудник, Ленинградский кораблестроительный институт, Ленинград, СССР}}{ivanov@cmc.msu.ru, petrov@cmc.msu.su}


Текст тезисов. Максимально допустимый объём тезисов --- не более
\textbf{двух страниц}, включая список литературы! Для тезисов
содержащих иллюстрации и/или таблицы допускается объём до трёх
страниц.

\begin{theorem}\label{ivanovTheorem}
Формулировка этой правильно оформленной теоремы содержит формулу

\begin{equation}\label{IvanovOne} % Метка начинается с фамилии автора
\left.\dfrac{df(t)}{dt}\right|_{t_0} = \lim\limits_{\Delta t \to 0} \dfrac{f(t_0 + \Delta t) - f(t_0)}{\Delta t},
\end{equation}
ссылки на которую могут быть оформлены с помощью стандартных средств \LaTeX.
\end{theorem}

На теорему можно сослаться с помощью команды \texttt{ref}. На формулу в Теореме \ref{ivanovTheorem} можно сослаться с помощью команды \texttt{eqref}: \eqref{IvanovOne}. Начинайте каждую метку с фамилии автора тезиса: ivanovTheorem, ivanovEq и так далее.


%
%   Иллюстрации, если они есть
%

\Pictures
%Следующая команда повторяется для каждой иллюстрации
\Picture{ivanov_01}{Подпись к рисунку. На рисунке подписаны оси.}{0.9}

Файл картинки должен лежать в том же каталоге, что и сам тезис. Имя файла должно содержать фамилию автора (или авторов), например, ivanov\_1.png.
Формат картинки --- .jpg, .png или .tiff. 

Напоминаем вам, что сборник будет напечатан в чёрно-белой печати в формате А5. 

%
%   Список литературы, если он есть
%
\begin{references}
\Source Васильев\,Ф.\,П. Методы оптимизации. М.:~МЦНМО, 2011.

\Source Чебунин\,И.\,В. Условия управляемости для уравнения 
        Риккати~// Дифференциальные уравнения. 2003. Т.\,39,
        \No\,12. С.\,1654--1661.

\Source \ENGLISH{Joachims\,T. Training linear SVMs in linear time // In
        Proceedings of the 12th ACM SIGKDD international
        conference on Knowledge discovery and data mining,
        New York, USA, 2006, P.\,217--226.}

\Source Страница конкурса   <<Интернет - математика>>: 

\url{http://imat-relpred.yandex.ru}
\end{references}

\end{thesis} % Сам тезис должен быть полностью помещен внутри окрежения thesis

 