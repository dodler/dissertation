\documentclass[usePics]{lomonosov}

\begin{thesis}


\Title{Анализ применимости методов анализа изображений в задаче предсказания рядов}{{Лян\,А.\,И.}}
\Author{Лян-Артем-Игоревич}{Студент}{Московский физико-технический институт (государственный университет)}{Москва}{Россия}{tntlagf93@mail.ru}


Данное исследование является частью диссертации магистра на тему: “Предсказание курсов акций при помощи методов машинного обучения”. Целью является исследовать возможность применения методов анализа изображений для предсказания курсов акций. Исследование было проведено при помощи языка R.

В качестве выборки были использованы данные по курсам акций компаний IBM, Oracle и Sberbank.

В качестве базовой модели была использована модель экспоненциального сглаживания. Для формирования изображения было задействовано 50 моделей с шагом параметра альфа равным 0.02.

В дальнейшем к полученному изображению применяется фильтр или выбирается строка посередине изображения. Подобная комбинация фильтра и модели позволяет получить границы движения целевого ряда снизу движении ряда вверх и наоборот. 

В дальнейшем планируется исследование других моделей для формирования изображения, а также других фильтров и методов обработки изображения. 


\end{thesis}