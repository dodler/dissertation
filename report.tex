В работе были исследованы существующие методы предсказания рядов. В их числе метод экспоненциального сглаживания, а также его вариации, предназначенные для адаптации модели к сезонным колебаниям и линейному росту. Кроме того, были исследованы такие модели как АРСС и ее обобщение - АРИСС. 

Было проведено знакомство с рядом методов машинного обучения, таких как линейная классификация, кластеризация, ассоциативные правила, а также некоторыми архитектурами нейронных сетей.

Автор также ознакомился с алгебраическим подходом предсказания временных рядов, постановкой задачи, способами решения, метрик качества полученных предсказаний. 