\documentclass{article}
\usepackage[left=3cm,right=1cm, 
top=2cm,bottom=2.5cm,bindingoffset=0cm]{geometry}
\usepackage[12pt]{extsizes}
\usepackage{amsmath, amssymb}
\usepackage{ucs}
\usepackage[utf8x]{inputenc} % Включаем поддержку UTF8
\usepackage[russian]{babel} 
\allowdisplaybreaks
\usepackage{indentfirst}
\setlength{\parindent}{5ex} % Примерно соответсвует 5 символам.

\usepackage[T2A]{fontenc} % Поддержка русских букв

\begin{document}

\begin{titlepage}
\begin{center}


\textsc{\Large Московский Физико-Технический Институт \\ (Государственный университет)}\\[1cm]

\textsc{\normalsize Факультет инноваций и высоких технологий \\ Кафедра банковских информационных технологий}\\[3cm]

{ \huge \bfseries Анализ временных рядов при помощи методов обработки изображений \\[3cm] }

\end{center}

\noindent\textnormal{\normalsize Выполнил: \\ cтудент 292 гр. \hfill \rule{3.6cm}{0.7pt} \hspace{1pt} Лян Артем Игоревич}\\[1cm]

\textnormal{\normalsize \\ Научный руководитель:  \hfill \rule{3.6cm}{0.7pt} \hspace{1pt} Филипенков Николай Владимирович}\\

\vspace{\fill} 

\begin{center}

\textnormal{Москва, 2016}

\end{center}

\thispagestyle{empty}
\end{titlepage}

\tableofcontents
\newpage

\section{ВВЕДЕНИЕ}
%\begin{center} \begin{LARGE}\textbf{ВВЕДЕНИЕ} \end{LARGE}\end{center}
\par{
Анализ временных рядов всегда был крайне актуальной задачей, стоявшей перед человечеством. Важность этой задачи была обусловлена накопленной статистической информацией и вместе с тем стимулировала развитие методов анализа этих данных. 
}

\par{
Временным рядом принято называть последовательность чисел, которые представляют собой значение некоторого процесса в дискетные моменты времени. Как правило, эти числа представляют собой значения процесса, измеренные через равные промежутки времени. 
}

\par{
Во многих прикладных задача возникает необходимочть исследования нескольких процессов одновременно. Тогда говорят о многомерном ряде, который определяется как последовательность векторов, содержащий значения нескольких показателей в один момент времени. Многомерный временной ряд также может быть представлен как совокупность временных рядов. 
}

\par{
В дальнейшем будут анализироваться именно многомерные временные ряды. 
}

\newpage

\section{Исследование методов предсказания рядов}
\par{
В работе были исследованы существующие методы предсказания рядов. В их числе метод экспоненциального сглаживания, а также его вариации, предназначенные для адаптации модели к сезонным колебаниям и линейному росту. Кроме того, были исследованы такие модели как АРСС и ее обобщение - АРИСС. \cite{lukashin2003} 
}

\par{
Экспоненциальное сглаживание - один из простейших и распространенных приемов выравнивания рядов. В его основе лежит расчет экспоненциальных средних. Экспоненциальное сглаживание ряда осуществляется по рекуррентной формуле
}
\begin{equation}
S_t = \alpha x_t + \beta S_{t-1}.
\end{equation}

\par{
Тут $$S_t$$ -  значение экспоненциальной средней в момент времени t , $$\alpha$$ - параметр сглаживания, $$\beta = 1 - \alpha$$. Величина $$S_t$$ является взвешенной суммой всех членов ряда. Значимость предыдущих членов ряда падает экспоненциально в зависимости от "возраста" наблюдения. \cite{lukashin2003} 
}

\par{
Было проведено знакомство с рядом методов машинного обучения, таких как линейная классификация, кластеризация, ассоциативные правила, а также некоторыми архитектурами нейронных сетей.
}

\newpage
\section{Постановка задачи}
\par{
Задача отнесения объекта к некоторым классам из заданного списка классов состоит в следующем. Дано множество М объектов, относительно которых производится классификация. Известно, что множество М представимо в виду суммы подмножеств $K_1,...,K_l$, называемых обычно классами. 
}

\par{
Задана информация $I$ о классах $K_1,...,K_l$, описание множества $M$ и описание $I(S)$ объекта $S$, о котором не известно - к какому из классов $K_1,...,K_l$ он принадлежит. Требуется по информации $I$, описанию $I(S)$ установить для каждого $j$ значение свойств $S \in K_j, j=1,2,...,l$. \cite{zhuravlev78}
}

\par{
Вышеуказанную задачу можно решать различными методами, среди которых - нейронные сети. Будем рассматривать часть временного ряда, как изображение. Будем подавать отрезок многомерного ряда на вход нейронной сети, обучать на этих данных и получать некоторую значение свойства $S$.
}

\par{
Кроме того, планируется рассмотреть возможность применения алгоритмов обработки изображений применительно к отрезку многомерного ряда. 
}

\newpage

%----------------------------------------------------------

\bibliographystyle{utf8gost705u}  %% стилевой файл для оформления по ГОСТу
\bibliography{biblio}     %% имя библиографической базы (bib-файла) 



\end{document}
